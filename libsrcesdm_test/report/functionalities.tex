\chapter{Functionalities}

\tab
This chapter compares the NetCDF functionalities with the current version of ESDM.

\section{Error Messages}

\tab
NetCDF has an extense classification for the possible errors that might happen. ESDM does not share this classification and it is something that their developers are not considering to include in the final version. This decision does not affect the performance of ESDM, but it was critical when the NetCDF tests were evaluated. Several wrong conditions are designed by NetCDF tests and ESDM does not produce the expected error. Because of that, the code in the tests that consider invalid parameters as input were removed.

\section{Classic Model/Modes}

\tab
There are other restrictions. For instance, NetCDF includes tests with the Classic Model, which ESDM does not support and it has no intention of doing it.

There are tests that consider in which mode the file is open. There are two modes associated with accessing a netCDF file: {\bf define mode} and {\bf data mode}. In define mode, dimensions, variables, and new attributes can be created but variable data cannot be read or written. In data mode, data can be read or written and attributes can be changed, but new dimensions, variables, and attributes cannot be created. The current version of ESDM does not have restrictions in the modes. Once the file is open, the user can do any modifications s/he wants.

Tables \ref{tab_modes_create} amd \ref{tab_modes_open} compares the options for creating and opening a file using NetCDF and ESDM.

\begin{table}[H]
\centering
\begin{tabular}{|l|m{6cm}|l|}
\hline
FLAG & NetCDF Support & ESDM Support \\ \hline \hline
NC\_CLOBBER & Overwrite existing file &  ESDM\_CLOBBER  (???)     \\ \hline
NC\_NOCLOBBER & Do not overwrite existing file &  ESDM\_NOCLOBBER  (???)      \\ \hline
NC\_SHARE & Limit write caching - netcdf classic files only &  NOT SUPPORTED       \\ \hline
NC\_64BIT\_OFFSET & Create 64-bit offset file &    NOT SUPPORTED     \\ \hline
NC\_64BIT\_DATA  & Create CDF-5 file (alias NC\_CDF5) &   NOT SUPPORTED      \\ \hline
NC\_NETCDF4 & Create netCDF-4/HDF5 file &  NOT SUPPORTED       \\ \hline
NC\_CLASSIC\_MODEL & Enforce netCDF classic mode on netCDF-4/HDF5 files &   NOT SUPPORTED      \\ \hline
NC\_DISKLESS & Store data in memory &    NOT SUPPORTED     \\ \hline
NC\_PERSIST & Force the NC\_DISKLESS data from memory to a file &  NOT SUPPORTED       \\ \hline
\hline
\end{tabular}
\caption{\label{tab_modes_create} Modes: creating a file.}
\end{table}

\begin{table}[H]
\centering
\begin{tabular}{|l|m{8cm}|l|}
\hline
FLAG & NetCDF Support & ESDM Support \\ \hline \hline
NC\_NOWRITE & Open the dataset with read-only access &  ESDM\_MODE\_FLAG\_READ       \\ \hline
NC\_WRITE & Open the dataset with read-write access &  ESDM\_MODE\_FLAG\_WRITE       \\ \hline
NC\_SHARE & Share updates, limit caching &  NOT SUPPORTED       \\ \hline
NC\_DISKLESS & Store data in memory &    NOT SUPPORTED     \\ \hline
NC\_PERSIST & Force the NC\_DISKLESS data from memory to a file &  NOT SUPPORTED       \\ \hline
\hline
\end{tabular}
\caption{\label{tab_modes_open} Modes: opening a file.}
\end{table}

\section{Datatypes}

\tab
ESDM supports the same basic types that NetCDF. The correspondence between the datatypes are provided in Tables \ref{basic-datatypes-size} and \ref{basic-datatypes-netcdf}.

\begin{table}[H]
\centering
\begin{tabular}{|l|c|l|l|}
\hline
NetCDF TYPE & Number & ESDM Type & ESDM Representation \\ \hline \hline
NC\_BYTE       &  1   & SMD\_DTYPE\_INT8     & int8\_t    \\ \hline
NC\_UBYTE      &  7   & SMD\_DTYPE\_UINT8    & uint8\_t    \\ \hline
NC\_CHAR       &  2   & SMD\_DTYPE\_CHAR     & char    \\ \hline
NC\_SHORT      &  3   & SMD\_DTYPE\_INT16    & int16\_t    \\ \hline
NC\_USHORT     &  8   & SMD\_DTYPE\_UINT16   & uint16\_t    \\ \hline
NC\_INT        &  4   & SMD\_DTYPE\_INT32    & int32\_t    \\ \hline
NC\_LONG       &  4   & SMD\_DTYPE\_INT32    & int32\_t    \\ \hline
NC\_UINT       &  9   & SMD\_DTYPE\_UINT32   & uint32\_t    \\ \hline
NC\_INT64      &  10  & SMD\_DTYPE\_INT64    & int64\_t    \\ \hline
NC\_UINT64     &  5   & SMD\_DTYPE\_UINT64   & uint64\_t    \\ \hline
NC\_FLOAT      &  11  & SMD\_DTYPE\_FLOAT    & 32 bits    \\ \hline
NC\_DOUBLE     &  6   & SMD\_DTYPE\_DOUBLE   & 64 bits    \\ \hline
\hline
\end{tabular}
\caption{\label{basic-datatypes-size} Equivalence between ESDM and NetCDF4 datatypes -- Datatypes sorted by size.}
\end{table}

\begin{table}[H]
\centering
\begin{tabular}{|l|c|l|l|}
\hline
NetCDF TYPE & Number & ESDM Type & ESDM Representation \\ \hline \hline
NC\_BYTE       &  1   & SMD\_DTYPE\_INT8     & int8\_t    \\ \hline
NC\_CHAR       &  2   & SMD\_DTYPE\_CHAR     & char    \\ \hline
NC\_SHORT      &  3   & SMD\_DTYPE\_INT16    & int16\_t    \\ \hline
NC\_INT        &  4   & SMD\_DTYPE\_INT32    & int32\_t    \\ \hline
NC\_LONG       &  4   & SMD\_DTYPE\_INT32    & int32\_t    \\ \hline
NC\_UINT64     &  5   & SMD\_DTYPE\_UINT64   & uint64\_t    \\ \hline
NC\_DOUBLE     &  6   & SMD\_DTYPE\_DOUBLE   & 64 bits    \\ \hline
NC\_UBYTE      &  7   & SMD\_DTYPE\_UINT8    & uint8\_t    \\ \hline
NC\_USHORT     &  8   & SMD\_DTYPE\_UINT16   & uint16\_t    \\ \hline
NC\_UINT       &  9   & SMD\_DTYPE\_UINT32   & uint32\_t    \\ \hline
NC\_INT64      &  10  & SMD\_DTYPE\_INT64    & int64\_t    \\ \hline
NC\_FLOAT      &  11  & SMD\_DTYPE\_FLOAT    & 32 bits    \\ \hline
\hline
\end{tabular}
\caption{\label{basic-datatypes-netcdf} Equivalence between ESDM and NetCDF4 datatypes -- Datatypes sorted by NETCDF description.}
\end{table}

The current version of ESDM does not support user-defined datatypes, but the developers intend to support this feature in the final version.

\begin{table}[H]
\centering
\begin{tabular}{|l|m{6cm}|l|}
\hline
FLAG & NetCDF Support & ESDM Support \\ \hline \hline
NC Data Type  &  NC Description  & ESDM Data Type  \\ \hline
NC\_NAT & NAT = Not A Type (c.f. NaN) &    SMD\_TYPE\_AS\_EXPECTED        \\ \hline
NC\_BYTE & signed 1 byte integer &     SMD\_DTYPE\_INT8        \\ \hline
NC\_CHAR & ISO/ASCII character &      SMD\_DTYPE\_CHAR       \\ \hline
NC\_SHORT & signed 2 byte integer &   SMD\_DTYPE\_INT16          \\ \hline
NC\_INT & signed 4 byte integer &     SMD\_DTYPE\_INT32        \\ \hline
NC\_LONG & deprecated, but required for backward compatibility &    SMD\_DTYPE\_INT32         \\ \hline
NC\_FLOAT & single precision floating point number &   SMD\_DTYPE\_FLOAT           \\ \hline
NC\_DOUBLE & double precision floating point number &   SMD\_DTYPE\_DOUBLE          \\ \hline
NC\_UBYTE & unsigned 1 byte int &     SMD\_DTYPE\_UINT8        \\ \hline
NC\_USHORT & unsigned 2-byte int &    SMD\_DTYPE\_UINT16         \\ \hline
NC\_UINT & unsigned 4-byte int &   SMD\_DTYPE\_UINT32          \\ \hline
NC\_INT64 & signed 8-byte int &    SMD\_DTYPE\_INT64         \\ \hline
NC\_UINT64 & unsigned 8-byte int &    SMD\_DTYPE\_UINT64         \\ \hline
NC\_STRING & string &    SMD\_DTYPE\_STRING         \\ \hline
NC\_VLEN & used internally for vlen types &      NOT SUPPORTED YET       \\ \hline
NC\_OPAQUE & used internally for opaque types &     NOT SUPPORTED YET        \\ \hline
NC\_COMPOUND & used internally for compound types &    NOT SUPPORTED YET         \\ \hline
NC\_ENUM & used internally for enum types &       NOT SUPPORTED YET      \\ \hline \hline
\end{tabular}
\caption{Datatypes Support.}
\end{table}

\section{Compression}

\tab
ESDM does not support compression yet. This means that all functions and tests related to chunking, deflate and fletcher will not work when using ESDM. There is a compression library read to be used, but that was not yet integrated in the current version of ESDM. The Scientific Compression Library (SCIL) can be found in the following Git Repository:

\begin{center}
\url{https://github.com/JulianKunkel/scil/}
\end{center}

\section{Endianness}

\tab
ESDM does not support native endianness.

\section{Groups}

\tab
In general, ESDM does not support groups from NetCDF. When there is one group without subgroups, ESDM can work properly and assumes the group and the file are the same entity. When there are more than one group or subgroups, ESDM can be called using the following command-line interface:

???

The ability to work with groups is a functionality that ESDM developers intend to have in the final version.

\begin{comment}

\begin{table}[H]
\centering
\begin{tabular}{|l|l|l|}
\hline
NetCDF ERROR                  &  Description                                                                        & ESDM ERROR \\ \hline\hline
NC\_EBADDIM                   &  Invalid dimension id or name.                                                      &      \\ \hline
NC\_EBADNAME                  &  Attribute or variable name contains illegal characters.                            &      \\ \hline
NC\_EHDFERR                   &  Error at HDF5 layer.                                                               &      \\ \hline
NC\_EINDEFINE                 &  Operation not allowed in define mode.                                              &      \\ \hline
NC\_EINVAL                    &  Invalid Argument.                                                                  &      \\ \hline
NC\_ELATEFILL                 &  Attempt to define fill value when data already exists.                             &      \\ \hline
NC\_ENOTINDEFINE              &  Operation not allowed in data mode.                                                &      \\ \hline
NC\_ENOTVAR                   &  Variable not found.                                                                &      \\ \hline
NC\_EPERM                     &  Write to read only.                                                                &      \\ \hline
NC\_ERANGE                    &  Math result not representable.                                                     &      \\ \hline
NC\_FILL\_FLOAT               &  Default fill value.                                                                &      \\ \hline
NC\_FORMAT\_64BIT\_OFFSET     &  Format specifier for nc\_set\_default\_format() and returned by nc\_inq\_format.   &      \\ \hline
NC\_FORMAT\_CLASSIC           &  Format specifier for nc\_set\_default\_format() and returned by nc\_inq\_format.   &      \\ \hline
NC\_FORMAT\_NETCDF4           &  Format specifier for nc\_set\_default\_format() and returned by nc\_inq\_format.   &      \\ \hline
NC\_FORMAT\_NETCDF4\_CLASSIC  &  Format specifier for nc\_set\_default\_format() and returned by nc\_inq\_format.   &      \\ \hline
\hline
\end{tabular}
\caption{Convertion between ESDM and NetCDF4 Erros.}
\end{table}

\end{comment}
