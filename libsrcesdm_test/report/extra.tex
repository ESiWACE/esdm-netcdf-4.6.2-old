Whether a range error occurs in writing a large floating-point value near the boundary of representable values may depend on the platform. The largest floating-point value you can write to a NetCDF float variable is the largest floating-point number representable on your system that is less than 2 to the 128th power. The largest double-precision value you can write to a double variable is the largest double-precision number representable on your system that is less than 2 to the 1024th power.

\begin{comment}

Tests with ESDM conversion:

\begin{verbatim}

Final Results - INT8 to FLOAT

********************************************
INT8_MAX = 127
Maximum FLOAT = 126.000000
Maximum INT8 = 126
********************************************

********************************************
INT8_MIN = -128
Minimum FLOAT = -127.000000
Minimum INT8 = -127
********************************************

Final Results - INT16 to FLOAT

********************************************
INT16_MAX = 32767
Maximum FLOAT = 32766.000000
Maximum INT16 = 32766
********************************************

********************************************
INT16_MIN = -32768
Minimum FLOAT = -32767.000000
Minimum INT16 = -32767
********************************************

Final Results - INT32 to FLOAT

********************************************
INT32_MAX = 2147483647
Maximum FLOAT = 2147483520.000000
Maximum INT32 = 2147483583
********************************************

********************************************
INT32_MIN = -2147483648
Minimum FLOAT = -2147483520.000000
Maximum INT32 = -2147483583
********************************************

Final Results - INT64 to FLOAT

********************************************
INT64_MAX = 9223372036854775807
Maximum FLOAT = 9223371487098961920.000000
Maximum INT64 = 9223371761976865807
********************************************

********************************************
INT64_MIN = -9223372036854775808
Minimum FLOAT = -9223371487098961920.000000
Minimum INT64 = -9223371761976865808
********************************************

Final Results - UINT8 to FLOAT

********************************************
UINT8_MAX = 255
Maximum FLOAT = 254.000000
Maximum UINT8 = 254
********************************************

Final Results - UINT16 to FLOAT

********************************************
UINT16_MAX = 65535
Maximum FLOAT = 65534.000000
Maximum UINT16 = 65534
********************************************

Final Results - UINT32 to FLOAT

********************************************
UINT32_MAX = 4294967295
Maximum FLOAT = 4294967040.000000
Maximum UINT32 = 4294967167
********************************************

Final Results - UINT64 to FLOAT

********************************************
UINT64_MAX = 18446744073709551615
Maximum FLOAT = 18446742974197923840.000000
Maximum UINT64 = 18446743523953731615
********************************************

Final Results - INT8 to DOUBLE

********************************************
INT8_MAX = 127
Maximum DOUBLE = 126.000000
Maximum INT8 = 126
********************************************

********************************************
INT8_MIN = -128
Minimum DOUBLE = -127.000000
Minimum INT8 = -127
********************************************

Final Results - INT16 to DOUBLE

********************************************
INT16_MAX = 32767
Maximum DOUBLE = 32766.000000
Maximum INT16 = 32766
********************************************

********************************************
INT16_MIN = -32768
Minimum DOUBLE = -32767.000000
Minimum INT16 = -32767
********************************************

Final Results - INT32 to DOUBLE

********************************************
INT32_MAX = 2147483647
Maximum DOUBLE = 2147483646.000000
Maximum INT32 = 2147483646
********************************************

********************************************
INT32_MIN = -2147483648
Minimum DOUBLE = -2147483647.000000
Minimum INT32 = -2147483647
********************************************

Final Results - INT64 to DOUBLE

********************************************
INT64_MAX = 9223372036854775807
Maximum DOUBLE = 9223372036854774784.000000
Maximum INT64 = 9223372036854775295
********************************************

********************************************
INT64_MIN = -9223372036854775808
Minimum DOUBLE = -9223372036854774784.000000
Minimum INT64 = -9223372036854775295
********************************************

Final Results - UINT8 to DOUBLE

********************************************
UINT8_MAX = 255
Maximum DOUBLE = 254.000000
Maximum UINT8 = 254
********************************************

Final Results - UINT16 to DOUBLE

********************************************
UINT16_MAX = 65535
Maximum DOUBLE = 65534.000000
Maximum UINT16 = 65534
********************************************

Final Results - UINT32 to DOUBLE

********************************************
UINT32_MAX = 4294967295
Maximum DOUBLE = 4294967294.000000
Maximum UINT32 = 4294967294
********************************************

Final Results - UINT64 to DOUBLE

********************************************
UINT64_MAX = 18446744073709551615
Maximum DOUBLE = 18446744073709549568.000000
Maximum UINT64 = 18446744073709550591
********************************************

\end{verbatim}

About conversion...

Actually that is what happens:
      if (nc_def_var(ncid, VAR1_NAME, NC_BYTE, 1, dimids, &varid)) ERR;
      if (nc_enddef(ncid)) ERR;
      if (nc_put_var_uchar(ncid, varid, uchar_out) != NC_ERANGE) ERR;

I found this in the documentation:
> BUT netcdf library returns NC_BYTE data as 8-bit signed
> integers (instead of 8-bit unsigned integers)
> so that NC_BYTE values go from 0 to 127 and then from -128 to -1
> instead of going from 0 to 255 as wanted.

That explains it, right?
https://www.unidata.ucar.edu/software/netcdf/docs_rc/group__variables.html#gac6e82a7c808f1e3c895616415ffa8e5d

The shit function assumes we convert the variable data *on the fly* while writing to the file…

As we are not converting the data type or anything here, we are not supporting this.
I reckon, we should check for any function which NC type does not match our expected type, write an error and abort.
I’m not intending to support this (ever)...

LR. Was this not the point of having the conversion inside smd? At least between the types we support? I also checked something in the documentation and they don’t do it between char and numbers.

The difference is that one is for ATTRIBUTES and one for VARIABLES, SMD is not involved in VARIABLES and hopefully never be as this is a performance problem.
We *might* be able to do part of this, in case COMPRESSION is turned on, as we then have to do a conversion anyway!
Document it this way.

\end{comment}

\begin{comment}

\begin{table}[H]
\centering
\begin{tabular}{|l|l|l|}
\hline
\multicolumn{1}{|c|}{NetCDF ERROR} & \multicolumn{1}{|c|}{Description} & \multicolumn{1}{|c|}{ESDM ERROR} \\ \hline\hline
NC\_EBADDIM                   &  Invalid dimension id or name.                                                      &      \\ \hline
NC\_EBADNAME                  &  Attribute or variable name contains illegal characters.                            &      \\ \hline
NC\_EHDFERR                   &  Error at HDF5 layer.                                                               &      \\ \hline
NC\_EINDEFINE                 &  Operation not allowed in define mode.                                              &      \\ \hline
NC\_EINVAL                    &  Invalid Argument.                                                                  &      \\ \hline
NC\_ELATEFILL                 &  Attempt to define fill value when data already exists.                             &      \\ \hline
NC\_ENOTINDEFINE              &  Operation not allowed in data mode.                                                &      \\ \hline
NC\_ENOTVAR                   &  Variable not found.                                                                &      \\ \hline
NC\_EPERM                     &  Write to read only.                                                                &      \\ \hline
NC\_ERANGE                    &  Math result not representable.                                                     &      \\ \hline
NC\_FILL\_FLOAT               &  Default fill value.                                                                &      \\ \hline
NC\_FORMAT\_64BIT\_OFFSET     &  Format specifier for nc\_set\_default\_format() and returned by nc\_inq\_format.   &      \\ \hline
NC\_FORMAT\_CLASSIC           &  Format specifier for nc\_set\_default\_format() and returned by nc\_inq\_format.   &      \\ \hline
NC\_FORMAT\_NETCDF4           &  Format specifier for nc\_set\_default\_format() and returned by nc\_inq\_format.   &      \\ \hline
NC\_FORMAT\_NETCDF4\_CLASSIC  &  Format specifier for nc\_set\_default\_format() and returned by nc\_inq\_format.   &      \\ \hline
\hline
\end{tabular}
\caption{Conversion between ESDM and NetCDF4 Errors.}
\end{table}

\end{comment}
